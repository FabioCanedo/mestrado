O Hamiltoniano do problema em questão é dado por:

\begin{multline}
H = \sum_{i=1}^{L} \epsilon_a a_{i}^\dagger a_{i} + \epsilon_b b_{i}^\dagger b_{i} - J \sum_{i=1}^{L} (a_{i+1}^\dagger a_i + a_{i}^\dagger a_{i+1}) - J \sum_{i=1}^{L} (b_{i+1}^\dagger b_i + b_{i}^\dagger b_{i+1}) \\ - J_{ab} \sum_{i=1}^{L} (a_i^\dagger b_i + b_i^\dagger a_i)
\label{ham}
\end{multline}

Para esse problema, começamos com a expansão:

\begin{equation}
a_j = \sum_{m=-\frac{L}{2}}^{\frac{L}{2}} e^{i(mj)} \tilde{a}_m
\label{fourier}
\end{equation}

Devido à condição de contorno $a_j=a_{j+L}$, teremos:

\begin{equation}
m = \frac{2 \pi n}{L} \text{ onde: } n \in \mathbb{Z}
\end{equation}

Multiplicando a equação \eqref{fourier} por $e^{-ijk}$ e somando em j de $1$ a $L$, obteremos:


\begin{align}
\begin{split}
\sum_{j=1}^{L} e^{-ijk} a_j &= \sum_{j=1}^{L} \sum_{m=-\frac{L}{2}}^{\frac{L}{2}} e^{-ijk} e^{i(mj)} \tilde{a}_m \\
&= \sum_{m=-\frac{L}{2}}^{\frac{L}{2}} \tilde{a}_m \sum_{j=1}^{L} e^{ij(m-k)} \\
&= \sum_{m=-\frac{L}{2}}^{\frac{L}{2}} \tilde{a}_m \delta_{m}^{k} L \\
&= L \tilde{a}_k \\
\end{split}
\end{align}

Portanto:

\begin{equation}
\tilde{a}_k = \frac{1}{L} \sum_{j=1}^{L} e^{-ijk} a_j
\end{equation}

Além disso, queremos que:

\begin{equation}
a_{j}^* = a_{j}
\end{equation}

Substituindo a Equação \eqref{fourier} e mudando a variável de soma do lado esquerdo($m \rightarrow -m$), obtemos:

\begin{equation}
\sum_{m=-\frac{L}{2}}^{\frac{L}{2}} e^{i(mj)} \tilde{a^{*}}_{-m} - e^{i(mj)} \tilde{a}_m = 0
\end{equation}

Essa soma somente vale zero se:

\begin{equation}
\tilde{a}_{m}^* = \tilde{a}_{-m}
\end{equation}

Agora, é possível demonstrar que:

\begin{subequations}
\begin{align}
\sum_{j=1}^{L} a_{j}^{\dagger} a_{j} &= L\sum_{m=-\frac{L}{2}}^{\frac{L}{2}} \tilde{a}_{m}^{\dagger} \tilde{a}_{-m} \\
\sum_{j=1}^{L} a_{j+1}^{\dagger} a_{j} &= L\sum_{m=-\frac{L}{2}}^{\frac{L}{2}} \tilde{a}_{m}^{\dagger} \tilde{a}_{-m} e^{im} \\
\sum_{j=1}^{L} a_{j}^{\dagger} a_{j+1} &= L\sum_{m=-\frac{L}{2}}^{\frac{L}{2}} \tilde{a}_{m}^{\dagger} \tilde{a}_{-m} e^{-im} \\
\sum_{j=1}^{L} a_{j}^{\dagger} b_{j} &= L\sum_{m=-\frac{L}{2}}^{\frac{L}{2}} \tilde{a}_{m}^{\dagger} \tilde{b}_{-m}
\end{align}
\end{subequations}

Substituindo essas identidades em \eqref{ham}, obtemos:

\begin{equation}
\begin{split}
H = L \sum_{m=-\frac{L}{2}}^{\frac{L}{2}} \bigg\{ 
\epsilon_a \tilde{a}_{m}^{\dagger} \tilde{a}_{-m} +
\epsilon_b \tilde{b}_{m}^{\dagger} \tilde{b}_{-m}
&-2J\cos(m)
\left(
\tilde{a}_{m}^{\dagger} \tilde{a}_{-m} +
\tilde{b}_{m}^{\dagger} \tilde{b}_{-m}
\right) \\
& -2J_{ab} \left(
\tilde{a}_{m}^{\dagger} \tilde{b}_{-m} +
\tilde{b}_{m}^{\dagger} \tilde{a}_{-m}
\right)
\bigg\}
\end{split}
\label{newham}
\end{equation}

Se definirmos as matrizes:

\begin{subequations}
\begin{align}
\tilde{c}_{m} :=&
\begin{bmatrix}
\tilde{a}_{m} \\
\tilde{b}_{m}
\end{bmatrix} \\
\tilde{c}_{m}^{\dagger} :=&
\begin{bmatrix}
\tilde{a}_{m}^{\dagger} &
\tilde{b}_{m}^{\dagger}
\end{bmatrix} \\
Q_{m} :=&
\begin{bmatrix}
\epsilon_a -2J\cos(m) & -2J_{ab} \\
-2J_{ab} & \epsilon_b -2J\cos(m)
\end{bmatrix}
\end{align}
\end{subequations}

Se substituirmos essas definições em \eqref{newham}, podemos reescrevê-lo como:

\begin{equation}
H = L \sum_{m=-\frac{L}{2}}^{\frac{L}{2}} \tilde{c}_{m}^{\dagger} Q_{m} \tilde{c}_{-m}
\end{equation}

Diagonalizando essa nova matriz $Q_{m}$ obteremos uma nova base como combinação linear dos operadores $\tilde{a}_m$, $\tilde{a}_m^{\dagger}$, $\tilde{b}_m$ e $\tilde{b}_m^{\dagger}$. Nessa base, o Hamiltoniano será diagonal, e teremos dois tipos de excitação diferentes da rede, representadas pelos dois autovetores dessa matriz, cada qual com sua relação de dispersão. Os autovalores dessa nova matriz serão:

\begin{subequations}
\begin{align}
\lambda_{\mu, m} &=  \frac{\epsilon_{a}+\epsilon_{b}}{2} + 2J\cos(m) - \sqrt{ 2J_{ab}^{2} + \left(\frac{\epsilon_{a} - \epsilon_{b}}{2}\right)^{2}} \\
\lambda_{\nu, m} &=  \frac{\epsilon_{a}+\epsilon_{b}}{2} + 2J\cos(m) + \sqrt{ 2J_{ab}^{2} + \left(\frac{\epsilon_{a} - \epsilon_{b}}{2}\right)^{2}}
\end{align}
\end{subequations}

Essas são, portanto, as relações de dispersão desse Hamiltoniano.