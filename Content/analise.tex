A análise dos dados será feita através de programas de \emph{jet clustering}, especificamente
o FastJet\cite{noauthor_fastjet_nodate}. As funções de distribuição de energia e ângulo serão
calculadas para uma série de eventos. Os algoritmos que são utilizados nesse programa são descrito
no Apêndice \ref{algoritmos}. Com o uso desses algoritmos, a análise será feita previamente localizando
os jatos nos dados. Em seguida, variando parâmetros como o $R$ ou parando em passos anteriores como no
caso do algoritmo $k_T$, podemos identificar subjatos dentro dos jatos. Após esse procedimento, é possível
medir abertura angular e fração energética dos jatos. Os histogramas montados para cada modelo
irão apresentar o comportamento dos observáveis em cada caso.