Abaixo, segue o cronograma planejado para a realização do trabalho:

\begin{tabular}{|c|p{8cm}|}
  \hline
  1º semestre de 2018 & Obtenção dos créditos do programa de pós-graduação do IFUSP \\
		      & Estudos introdutórios da área de íons pesados relativísticos \\ \hline
  2º semestre de 2018 & Obtenção dos créditos do programa de pós-graduação do IFUSP \\
		      & Estudo e contextualização do problema a ser investigado: \\
		      & \tabitem Levantamento bibliográfico sobre os processos de produção de quarks pesados; \\
		      & \tabitem Estudo de alguns geradores de evento de colisões hadrônicas, com ênfase nos processos de perda de energia dos quarks pesados; \\
		      & \tabitem Estudo dos algoritmos de reconstrução de jatos, bem como dos algoritmos que podem realizar análise de subestrutura dos jatos. \\ \hline
  1º semestre de 2019 & Simulação de eventos com os modelos e geradores estudados previamente \\
  		      & Início da análise dos dados \\ \hline
  2º semestre de 2019 & Finalização da análise dos dados simulados \\
  		      & Redação da dissertação \\ \hline
\end{tabular}
