Para a realização desse trabalho utilizaremos geradores de eventos, que são essencialmente simuladores
de colisões de íons pesados, para verificar os efeitos esperados das fragmentações de quarks pesados
nos observáveis de subestrutura de jatos. Os programas que serão utilizados serão:

\begin{itemize}
 \item MUSIC\cite{noauthor_music_nodate}, que possui uma breve descrição na subseção \ref{music};
 \item PYTHIA\cite{noauthor_pythia_nodate}, que constitui em um gerador de jatos, especificamente;
 \item HYDJET\cite{lokhtin_hydjet++_2009}, constitui um gerador de jatos combinado com um algoritmo de simulação hidrodinâmica, ver subseção \ref{hjet};
 \item Jewel\cite{noauthor_jewel_nodate, zapp_jewel_2014}, constitui um gerador de eventos com modelos específicos de \emph{Jet Quenching};
\end{itemize}
