No início de colisões de íons pesados, quarks pesados podem ser gerados através dos mecanismos mostrados na Figura \ref{criacao}.

\begin{figure}[!hbt]
\begin{floatrow}
 \subfloat[Criação de par através de aniquilação de gluons.]{
 \begin{tikzpicture}[scale=.45]
  \draw [gluon] (0,3) -- (4,2.5) node[near start,anchor=south,red]{$g$};
  \draw [gluon] (0,0) -- (4,.5) node[near start,anchor=north,red]{$g$};
  \draw [afermion] (4,2.5) -- (8,3) node[near end,anchor=south,red]{$\overline{Q}$};
  \draw [fermion] (4,2.5) -- (4,.5);
  \draw [fermion] (4,.5) -- (8,0) node[near end,anchor=south,red]{$Q$};
 \end{tikzpicture}
 \label{diagramaa}
 }
 
 \subfloat[Criação de par de quarks pesados através de aniquilação de quarks leves oriundos dos núcleons iniciais.]{
 \begin{tikzpicture}[scale=.35]
  \draw [fermion] (0,4) -- (3,2) node[near start,anchor=south,red]{$q$};
  \draw [afermion] (0,0) -- (3,2) node[near start,anchor=north,red]{$\overline{q}$};
  \draw [gluon] (3,2) -- (5,2);
  \draw [fermion] (5,2) -- (8,4) node[near end,anchor=south,red]{$Q$};
  \draw [afermion] (5,2) -- (8,0) node[near end,anchor=north,red]{$\overline{Q}$};
 \end{tikzpicture}
 \label{diagramab}
 }
 
 \subfloat[Mesmo processo do caso anterior, mas com emissão de gluon posterior.]{
 \begin{tikzpicture}[scale=.45]
  \draw [gluon] (0,3) -- (4,2.5) node[near start,anchor=south,red]{$g$};
  \draw [gluon] (0,0) -- (4,.5) node[near start,anchor=north,red]{$g$};
  \draw [afermion] (4,2.5) -- (8,3) node[near end,anchor=south,red]{$\overline{Q}$};
  \draw [fermion] (4,2.5) -- (4,.5);
  \draw [fermion] (4,.5) -- (8,0) node[near end,anchor=north,red]{$Q$};
  \draw [gluon] (5,.375) -- (8,1.5) node[near end,anchor=south,red]{$g$};
 \end{tikzpicture}
 \label{diagramac}
 }
 
 \end{floatrow}
 
 \begin{floatrow}
  \subfloat[]{
 \begin{tikzpicture}[scale=.45]
  \draw [gluon] (0,3) -- (4,2.5) node[near start,anchor=south,red]{$g$};
  \draw [fermion] (4,2.5) -- (8,3) node[near end,anchor=south,red]{$Q$};
  \draw [afermion] (4,2.5) -- (5,2);
  \draw [gluon] (0,0) -- (4,0) node[near start,anchor=north,red]{$g$};
  \draw [gluon] (4,0) -- (5,2);
  \draw [gluon] (4,0) -- (8,0) node[near end,anchor=north,red]{$g$};
  \draw [afermion] (5,2) -- (8,2) node[near end,anchor=north,red]{$\overline{Q}$};
 \end{tikzpicture}
 }
 
 \subfloat[]{
 \begin{tikzpicture}[scale=.45]
  \draw [gluon] (0,3) -- (3,2.5) node[near start,anchor=south,red]{$g$};
  \draw [gluon] (3,2.5) -- (5,2.5);
  \draw [gluon] (0,0) -- (3,0)node[near start,anchor=north,red]{$g$};
  \draw [gluon] (3,0) -- (3,2.5);
  \draw [gluon] (3,0) -- (8,0) node[near end,anchor=north,red]{$g$};
  \draw [afermion] (5,2.5) -- (8,2) node[near end,anchor=north,red]{$\overline{Q}$};
  \draw [fermion] (5,2.5) -- (8,3) node[near end,anchor=south,red]{$Q$};
 \end{tikzpicture}
 }
 
 \subfloat[]{
 \begin{tikzpicture}[scale=.45]
  \draw [gluon] (0,4.5) -- (2,3.5) node[near start,anchor=south,red]{$g$};
  \draw [afermion] (2,3.5) -- (8,6.5) node[near end,anchor=south,red]{$\overline{Q}$};
  \draw [fermion] (2,3.5) -- (3,2.5);
  \draw [fermion] (3,2.5) -- (8,4.5) node[near end,anchor=south,red]{$Q$};
  \draw [gluon] (3,2.5) -- (4,1.5);
  \draw [gluon] (4,1.5) -- (8,3.5) node[near end,anchor=north,red]{$g$};
  \draw [gluon] (4,1.5) -- (4,0);
  \draw [gluon] (0,0) -- (4,0) node[near start,anchor=north,red]{$g$};
  \draw [gluon] (4,0) -- (8,0) node[near end,anchor=north,red]{$g$};
 \end{tikzpicture}
 }
 
 \end{floatrow}

 
 \caption{Processos de criação de quarks pesados.}
 \label{criacao}
\end{figure}


Quarks pesados, como o \emph{bottom}, podem ser utilizados como ponta de prova para o estudo do Plasma de Quarks e Gluons
devido à sua interação com o este em seu caminho para fora da região de interação\cite{li_inverting_2017,renk_jet_2014}. Eles
interagem com o meio de duas formas, através da radiação induzida, ou \emph{gluonsstrahlung}, e através de reações colisionais.
Embora estes mesmos feitos ocorram para quarks leves, a massa dos quarks pesados limita a sua perda de energia e também sua
velocidade\footnote{Especialmente no limite $E \approx m$}, o que permite que ele ``colete'' mais informações sobre o QGP.
\par
Os efeitos do meio podem ser especialmente observados na subestrutura dos jatos, que são causadas por funções de fragmentação.
Estas, por sua vez, estão diretamente relacionadas com os processos de perda de energia do parton no QGP. As funções de fragmentação
são distribuições de probabilidade que refletem processos do tipo $a \longrightarrow b+c$. Para um certo número $z \in [0,1]$, teremos
a relação entre as energias:

\begin{equation}
 E_a = z E_b + (1-z)E_c
\end{equation}

Podemos então, definir a função densidade de probabilidade do processo:

\begin{equation}
 P_{a \longrightarrow b+c}(z) = \frac{dN_{a \longrightarrow b+c}}{dz}
 \label{probz}
\end{equation}

Essa função pode ser medida na subestrutura dos jatos, assim como a equivalente em abertura angular:

\begin{equation}
 P_{a \longrightarrow b+c}(\theta) = \frac{dN_{a \longrightarrow b+c}}{d\theta}
 \label{probt}
\end{equation}

Essas quantidades podem ser calculadas com a utilização da pQCD\footnote{\emph{perturbative Quantum Chromodynamics}}, um exemplo desses
cálculos pode ser encontrado em \cite{seymour_jet_1998}. Esses efeitos, como supramencionado, podem ser observados na subestrutura dos
jatos. Por sua vez, a subestrutura pode ser observada através de variações no parâmetro $R$ nos algoritmos de reconstrução de jatos(ver
Apêndice \ref{algoritmos}. Ao diminuirmos esse parâmetro, os algoritmos evitam o \emph{merge} dos jatos e podemos então, para dois
jatos próximos comparar o a distância angular de dois jatos($\Delta R$) e a fração de energia que o jato mais leve possui da soma total.
Isso deve fornecer evidência direta das funções de probabilidade definidas nas equações \eqref{probz} e \eqref{probt}.