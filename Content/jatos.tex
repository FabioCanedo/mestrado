\subsection{Algoritmos de Reconstrução de Jatos}

Para realizar a reconstrução de jatos\cite{salam_towards_2010}, costuma-se definir a seguinte quantidade:

\begin{equation}
 \Delta R_{ij} = \sqrt{(\eta_i-\eta_j)^2+(\phi_i-\phi_j)^2}
\end{equation}

Essa será, para as duas partículas, aqui chamadas de $i$ e $j$, uma distância angular
definida entre as duas. A maioria dos algoritmos de jatos irão utilizar essa definição de distância angular para realizar os agrupamentos,
ou {\it clusters} de partículas para a reconstrução dos jatos. Estes são chamados os algoritmos de cones. Um algoritmo
comumente utilizado consiste em, primeiro definir um parâmetro $R$, em seguida, escolher uma partícula {\it semente}, normalmente, escolhe-se
a partícula de maor $p_T$. Após esses passos, localizamos a partícula de maior $p_T$ a uma distância menor ou igual a $R$.
\par
Então, retiramos essas duas partículas dos dados e definimos uma nova partícula com momento:

\begin{equation}
 p_{\mu}^{J} = p_{\mu}^1 + p_{\mu}^2
\end{equation}

Essa partícula define o novo centro do cone, agora procuramos a partícula de maior $p_T$ de distância menor ou igual a $R$ desta, e assim por
diante.
