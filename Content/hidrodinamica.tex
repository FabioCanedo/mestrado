Para descrever a evoluão do QGP, utiliza-se o formalismo da hidrodinâmica relativística. Algumas quantidades são
importantes de serem definidas:

\begin{equation}
 \frac{dx^{\mu}}{d\tau} = u^{\mu}(x)
\end{equation}

Essa será a velocidade, ou fluxo do fluido em cada elemento do espao-tempo. Também teremos o tensor energia-momento:

\begin{equation}
 T_{\mu \nu} = (\epsilon + P) u_\mu u_\nu - P g_{\mu \nu}
\end{equation}

Se aplicarmos a equação de conservação de energia-momento, obtemos quatro equações que descrevem a evoluao do fluido:

\begin{equation}
 \partial^\mu T_{\mu \nu} = 0
\end{equation}

Entretanto, temos quatro variáveis $u^{\mu}$ e mais duas $\epsilon$ e $P$, totalizando seis. Para resolvermos o sistema precisamos
de duas equações adicionais, são estas a normalização do quadrivetor($u^{\mu}u_{\mu}=1$ e a equação de estado:

\begin{equation}
 \epsilon=\epsilon(P)
\end{equation}


