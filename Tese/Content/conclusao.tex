The results presented in the previous chapter show that, as far as the jet structure and shape are concerned, there are no major modifications of the observables due to the inclusion of realistic hydro and initial conditions. Some improvements came on the jet mass, bit only slightly.
\par
To improve the description of the data, other things could be modified. First, there is still the hadronization mechanism that is used by JEWEL, which comes from PYTHIA. This mechanism is built to explain pp data. Jets formed in heavy-ion collisions might be substantially different due to their hadronization mechanism. Effects of coalescence might play a big role here, giving new color partners to the leading partons describing the jets.
\par
Another thing that was not implemented in this work is the local four-velocity and a realistic EOS. This could further improve the description of the data. Since the partons are moving through the medium and the medium itself is most of the time moving outwards, the collisions could be more collinear than JEWEL default predicts. This could collimate the jet further, resulting in narrower jets. The collisions would be softer as well, since the CM of the local elastic collisions would be smaller, perhaps resulting in harder hadrons in the center of the jets in the final state. This could improve the $p_T^D$ data as well.
\par
The sample of the results presented in this work shows that there is not, as far as internal jet substructure and shape is concerned, a major improvement in describing the observables by implementing more realistic initial conditions or realistic hydrodynamics. Also, the correlation between medium and jets does present modifications due to the inclusion of a more realistic background, as indicated in the $v_2$ results. There is still high uncertainty in the experimental data, and further experimental improvement is needed for more detailed comparisons. The fact that this observable showed a value different from zero does indicate that high energy partons can be used to investigate properties of the medium through this kind of observable. Further information can be potentially extracted from the medium by looking at higher moments of the Fourier decomposition. This can aid in constraining the models applied in these predictions, as well as extract the transport coefficients from the QGP.