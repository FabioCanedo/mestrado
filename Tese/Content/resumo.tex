\chapter*{Resumo}

Neste trabalho n\'{o}s investigamos poss\'{i}veis impactos que o plasma de Quarks e Gl\'{u}ons pode ter nos observ\'{a}veis de Jatos. N\'{o}s escolhemos o JEWEL (Jet Evolution With Energy Loss) para este estudo. N\'{o}s acoplamos o JEWEL com o modelo $\rm T_RENTo$ e tamb\'{e}m com o MC-KLN+vUSPhydro para as simula\c{c}\~{o}es. As simulaç\~{o}es foram realizadas para colis\~{o}es chumbo-chumbo a energia $\sqrt{s_{NN}} = 2.76 \, {\rm TeV}$ para centralidade $0-10\%$. Nessas condi\c{c}\~{o}es, observ\'{a}veis de forma e geometria dos jatos n\~{a}o s\~{a}o modificados pela implementa\c{c}\~{a}o de uma hidrodin\^{a}mica e condi\c{c}\~{o}es iniciais realistas. Tamb\'{e}m calculamos o $v_2$ dos jatos. Neste caso n\'{o}s conclu\'{i}mos que as condi\c{c}\~{o}es iniciais tamb\'{e}m n\~{a}o afetam esse observ\'{a}vel. No caso da hidrodin\^{a}mica realista, houve uma melhoria na descri\c{c}\~{a}o desse observ\'{a}vel.

\noindent\textbf{Palavras-chave:} \'{I}ons Pesados; Plasma de Quarks e Gluons; Hidrodin\^{a}mica; Supress\~{a}o de Jatos; Cromodin\^{a}mica Qu\^{a}ntica.