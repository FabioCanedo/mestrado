\chapter*{Agradecimentos}

Agrade\c{c}o o meu orientador, Marcelo G. Munhoz, pelo suporte e pelas discuss\~{o}es ao longo desse per\'{i}odo. Agrade\c{c}o tamb\'{e}m pela paci\^{e}ncia com meu estilo ca\'{o}tico de trabalho. Muitas de nossas conversas trouxeram-me grande paz de esp\'{i}rito e conforto para a minha ansiedade. Oxal\'{a} continuemos trabalhando juntos por bastante tempo. Agrade\c{c}o a Jacquelyn Noronha-Hostler e o Jorge Noronha pelas discuss\~{o}es. Agrade\c{c}o também os membros do HEPIC, em especial Cristiane Jahnke, Geovane Grossi, Henrique Zanoli, Lucas Teixeira e Ricardo Pitta. V\'{a}rios bugs n\~{a}o teriam sido resolvidos sem as minhas conversas com voc\^{e}s. Agrade\c{c}o o Ricardo Rom\~{a}o da Silva pelo apoio tamb\'{e}m em v\'{a}rios momentos com o SAMPA.

Agrade\c{c}o aos professores que tive durante a minha vida, que estimularam minha sede por f\'{i}sica e por conhecimento em geral. Nossos encontros fizeram de mim uma pessoa melhor. Em especial os professores Jailton Ara\'{u}jo Oliveira e S\'{e}rgio Souza. Agrade\c{c}o os colegas que tive tamb\'{e}m durante a gradua\c{c}\~{a}o, em especial a Carol Guandalim e o Victor Gomes Da Costa Lobo.

Agrade\c{c}o a minha companheira Malu, pelo carinho e amor. Pelos \'{o}timos momentos que tivemos juntos, pelos conselhos que me deu nessa jornada e por suportar minhas presepadas. Agrade\c{c}o tamb\'{e}m pelo apoio que me deu na revis\~{a}o deste trabalho. Agrade\c{c}o meus pais Vitor e Marcia por tudo. Agrade\c{c}o tamb\'{e}m o suporte e amor incondicional que me deram e continuam me dando. Agrade\c{c}o meus irm\~{a}os Felipe e Mikael por suportarem com gra\c{c}a, na maioria das vezes, minhas brincadeiras. Agrade\c{c}o ao Jacinto, meu salsicho, pelo calor e carinho enquanto redigi esse trabalho. Agrade\c{c}o os meus grandes amigos Gabriel Zoha e Rayner Ribeiro pelas discuss\~{o}es sobre f\'{i}sica, matem\'{a}tica e pela espor\'{a}dica companhia alc\'{o}lica.

Enfim, agrade\c{c}o ao CNPq, pelo aux\'{i}lio financeiro, sob o processo 132927/2018-7, sem o qual essa disserta\c{c}\~{a}o n\~{a}o poderia ter sido realizada.