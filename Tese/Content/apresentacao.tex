One of the key problems in particle physics today is the understanding of confinement, property of the fundamental particles that interact through the strong force. This property was originally hypothesized to explain the fact that no particles of fractional charge were ever observed in accelerators and cosmic rays experiments and it was a necessity of the quark model\cite{halzen_quarks_1984}. When the quark model was combined with QCD(\qcd), it became a major puzzle to explain this property in light of this theory.
\par
Jets are a consequence of the confinement property of matter. In accelerator experiments, two partons might interact by exchanging a large momentum, a phenomenon known as hard scattering. When this happens and the quarks start to increase their distance from the interaction point, their potential energy starts to grow linearly. This is different from electromagnetism in which the energy falls as $r^{-2}$. Due to this increase in energy, it eventually becomes favorable for the system to excite a pair of quarks from vacuum\cite{dissertori_quantum_2003}. The system evolves into a new pair of color neutral partons, which are closer together. This process occurs iteratively until the resulting pairs don't have enough energy to increase their distance. The result, experimentally, is a spray of particles that hit the detector with similar angles. These are called jets.
\par
The understanding of confinement necessarily requires also the understanding of QCD vacuum properties and it was pointed out by T. D. Lee\cite{lee_abnormal_1975}:

\begin{quote}
To study the question of ``vacuum", we must turn to a different direction, we should investigate some ``bulk" phenomena by distributing high energy over a relatively large volume.
\end{quote}

The excited state of matter, in which quarks and gluons are not confined, is called Quark-Gluon Plasma. This state of matter can be created in the laboratory today through ultra-relativistic heavy-ion collisions. The idea is that during the collision, a good part of the kinetic energy is converted into thermal energy, and a phase transition occurs. Later on, the matter expands and cools, transitioning back into hadrons, which are bound states of quarks and gluons. These final hadrons eventually stop interacting as well and start their travel as free particles. They potentially go through further decays of electromagnetic or weak nature and then hit the detectors.
\par
Once the particles hit the detector, they might get identified and their properties measured, such as their transverse momentum and their rapidity. The distribution of these particles is then analyzed to extract what has happened during the collision. One phenomenon that was identified in the study of this final state is the so-called Jet Quenching. It corresponds to the effects of the medium present in heavy-ion collisions on the hard scattering. When hard scattering occurs in heavy-ion collisions, the partons are surrounded by a color excited medium. Like in an ordinary plasma, a screening effect prevents their potential energy to grow linearly. They interact further with the hot and dense medium created in their surroundings before experience the process of jet creation. They usually lose energy due to elastic scattering and \emph{gluonstrahlung}, resulting in jets with less energy and broader.
\par
In this work, a study was performed of observables related to Jet Quenching that could be sensitive to the finer details of the initial conditions as well as the hydrodynamic evolution of the quark-gluon plasma formed in ultra-relativistic heavy-ion collisions. Several known observables were analyzed, both from the perspective of the inner jet substructure and shape, as well as soft-high $p_T$ correlation observables such as the jet $v_2$. Since that involves collective behavior associated with the soft sector.
\par
In Chapter \ref{theory}, the theory and the main models used in this work are explained. In Chapter \ref{method}, the observables, as well as the techniques used to extract and analyze them are explained. In Chapter \ref{results}, the results of the work are presented. In Chapter \ref{conclusions}, the conclusions and discussion of the results are presented.
%We have found in this work that the inclusion of a realistic hydrodynamic evolution, alongside the sophisticated initial condition models, do not change significantly the observables related to the substructure and jet shape variables. We have also found that $v_2$ is modified by the presence of these details.
%\par
%A good follow up of this work would be to investigate systematically how exactly are the $v_2$ affected by the parameters related to initial conditions and hydro evolution, such as viscosity and fluctuations, etc. An interesting suggestion is to study the response of other jet quenching models and hydro codes to check the consistency of these results to model dependency.