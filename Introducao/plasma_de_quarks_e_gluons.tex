\subsection{Plasma de Quarks e Gluons}

\ \ Em colisões íons pesados, ou seja, íons com número de massa da ordem de $10^2$, a uma energia da ordem de $10^2 GeV$,
uma quantidade considerável de energia é depositada na região de interação. Essa nergia, na forma de quarks e gluons,
libera novos graus de liberdade, realizando uma transição de fase para um estado da matéria conhecido como Plasma de Quarks e Gluons(QGP,
sigla em inglês).
\par
A temperatura necessária para formar este estado da matéria é da ordem de centenas de $MeV$ ou
$10^{12} K$, cerca de $10$ mil vezes a temperatura do centro do Sol, e a densidade de energia é da ordem de $0.2-1
GeV/fm^{3}$. As propriedades deste estado da matéria podem ser estudadas analisando os produtos dessa colisão após o resfriamento
da matéria. O espectro de $p_T$\footnote{Ver seção \ref{variaveis}} das partículas, por exemplo, fornece insformações sobre a entropia
e a temperatura do plasma, através da multiplicidade e da inclinação do gráfico, respectivamente. Em geral, essas propriedades estarão
associadas ao espectro na faixa de $p_T \approx 0-2 GeV/c$. Na faixa $p_T > 2 GeV/c$, observa-se os efeitos de fenômenos
da classe {\it hard scaterring}. Estes fenômenos são resultados da formação de partículas de alta energia que atravessam o plasma
aquecido, depositando energia neste. Na sua saída, devido às propriedades da QCD\footnote{QCD ou {\it Quantum Chromodynamics} é a 
teoria que descreve as interações fortes.}, essas partículas se fragmentam criando os chamados {\it jets}.



