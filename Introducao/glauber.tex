\subsection{Modelo de Glauber}\label{glauber}

Sempre que estudamos colisões de íons pesados, é necessário fornecer as condições iniciais da colisão. O modelo
de Glauber baseia-se na ideia de que os nucleons pertencentes ao núcleo projétil realizam colisões dois a dois seguindo trajetórias
retas atravessando o núcleo alvo. A distribuição dos nucleons nos dois núcleos, tanto alvo quanto o projétil, seguem a distribuição de
Woods-Saxon:

\begin{equation}
 \rho(r) = \frac{\rho_0}{1+\exp{(\frac{r-R}{a})}}
\end{equation}

Normalmente, os núcleons são gerados em uma distribuição espacial que obedece tal distribuição, em seguida, sua trajetória
é traçada em linha reta, considerando colisões com todos os núcleons em seu caminho. Uma descrição detalhada deste modelo e a
geração de condições iniciais pode ser encontrada em \cite{miller_glauber_2007}.