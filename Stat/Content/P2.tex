Para esse problema, teremos uma Hamiltoniana definida por:

\begin{equation}
H=kS_{x}^{2} + \lambda S_z
\end{equation}

Onde:

\begin{equation}
Sx=\left(
\begin{array}{ccc}
 0 & \frac{1}{\sqrt{2}} & 0 \\
 \frac{1}{\sqrt{2}} & 0 & \frac{1}{\sqrt{2}} \\
 0 & \frac{1}{\sqrt{2}} & 0 \\
\end{array}
\right),
Sz= \left(
\begin{array}{ccc}
 1 & 0 & 0 \\
 0 & 0 & 0 \\
 0 & 0 & -1 \\
\end{array}
\right)
\end{equation}

A Hamiltoniana escrita em forma de matriz fica então:

\begin{equation}
H= \left(
\begin{array}{ccc}
 \lambda & \frac{k}{2} & 0 \\
 \frac{k}{2} & 0 & \frac{k}{2} \\
 0 & \frac{k}{2} & -\lambda \\
\end{array}
\right)
\end{equation}

Os autovalores podem ser achados pelas raízes do polinômio característico, e serão:

\begin{equation}
\begin{split}
E_1&=0 \\
E_2&=-\sqrt{\frac{k^2}{2}+\lambda^2} \\
E_3&=\sqrt{\frac{k^2}{2}+\lambda^2}
\end{split}
\end{equation}

