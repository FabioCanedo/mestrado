Estudamos aqui o caso do qutrit. Trata-se de um sistema com três estados de energia, a saber:

\begin{equation}
\begin{split}
E_1 & =0 \\
E_2 & =\epsilon+\Delta \\
E_2 & =\epsilon-\Delta
\end{split}
\end{equation}

A quantidade mais importante para estudar as propriedades de equilíbrio desse sistema é 
a função de equipartição, definida por:

\begin{equation}
Z=\sum_i e^{-\beta E_i}
\end{equation}

A partir dessa, todas as outras quantidades de interesse podem ser extraídas. Aplicando, então, as energias supramencionadas, obtemos:

\begin{equation}
Z=1+2e^{-\beta \epsilon}\cosh(\beta \Delta)
\end{equation}

Um plot dessa função pode ser observada na Figura \ref{Z}

\begin{figure}
\includegraphics[scale=.3]{Z.png}
\caption{Função equipartição como função da temperatura em unidades de $\epsilon$.}
\label{Z}
\end{figure}

Para obtermos agora, como outras quantidades termodinâmicas se comportam, utilizamos, por exemplo, a identidade:

\begin{equation}
U=-\frac{\partial \ln Z}{\partial \beta}
\end{equation}

De onde podemos obter:

\begin{equation}
U=\frac{2(\epsilon \cosh(\beta \Delta) - \Delta \sinh(\beta \Delta))}{e^{\beta \epsilon} + 2 \cosh(\beta \Delta)}
\end{equation}

A forma dessa função, a chamada energia interna, pode ser observada na Figura \ref{U}

\begin{figure}[!h]
\includegraphics[scale=.3]{U0a10}
\caption{Energia Interna como função da temperatura do sistema calculada em unidades de $\epsilon$.}
\label{U}
\end{figure}

A partir da energia interna, é possível obter também a capacidade térmica:

\begin{equation}
C=\frac{2 \beta^2 \left(e^{\beta \epsilon} \left(\left(\Delta^2+e^2\right) \cosh (\beta \Delta)-2 \Delta \epsilon \sinh (\beta \Delta)\right)+2 \Delta^2\right)}{\left(2 \cosh (\beta \Delta)+e^{\beta \epsilon}\right)^2}
\end{equation}

A forma dessa última pode ser observada na Figura \ref{C}.

\begin{figure}[!h]
\includegraphics[scale=.3]{Cpeq}
\caption{Capacidade Térmica como função da temperatura do sistema calculada em unidades de $\epsilon$.}
\label{C}
\end{figure}

Para $T \rightarrow \infty$, teremos a capacidade térmica indo para zero. Isso está consistente com as características do sistema, para ilustrar esse ponto, basta pensarmos nas probabilidades de ocupação de cada estado, definidas por:

\begin{equation}
P_i=\frac{e^{-\beta E_i}}{Z}
\end{equation}

Como $T \rightarrow \infty \Rightarrow \beta \rightarrow 0$, teremos todas as probabilidades iguais a $\frac{1}{Z}$, isso resulta em estados equiprováveis e a uma energia igual à média das energias dos estados. Isso implica que a energia interna tem um limite finito bem definido, portanto, fica cada vez mais difícil acrescentar energia no sistema, de maneira que a capacidade térmica tende a zero.

\par

Outras duas quantidades de interesse a ser calculadas são a enerfia livre de Helmholtz e a entropia do sistema, $F$ e $S$, respectivamente. Estas são dadas pelas expressões:

\begin{equation}
\begin{split}
F & = \frac{\ln(Z)}{\beta} \\
S & = \beta(U + F)
\end{split}
\end{equation}

Ambas estão plotadas nas Figuras \ref{F} e \ref{S}.

\begin{figure}[!h]
\includegraphics[scale=.3]{F0a1}
\caption{Energia Livre de Helmholtz como função da temperatura do sistema calculada em unidades de $\epsilon$.}
\label{F}
\end{figure}

\begin{figure}[!h]
\includegraphics[scale=.3]{entropy}
\caption{Entropia como função da temperatura do sistema calculada em unidades de $\epsilon$.}
\label{S}
\end{figure}

É interessante notar que, para valores mais altos de $\Delta$, a entropia possui um pico que fica maior e se desloca para a esquerda. Para $\Delta=\epsilon$ a entropia do sistema à temperatura zero na realidade é diferente de 0. Isso ocorre porque, nesse caso, o estado fundamental é degenerado e teremos $P_1=P_3=0.5$ e zero para o outro estado. Utilizando a definição de entropia:

\begin{equation}
S=-\sum_i P_i \ln P_i
\end{equation}

Portanto, para $T \rightarrow 0$, teremos $S\rightarrow\ln(2)$