Nesse problema, começamos com o funcional assumindo que os campos são uniformes no espaço, isso resulta em:

\begin{equation}
F[\phi(x),A(x)] = C \left( \frac{a}{2} |\phi|^2 + \frac{b}{4} |\phi|^4 +  \lambda |\phi|^2 A^2 \right)
\end{equation}

Agora, derivando em relação a $A$ e $\phi$, obtemos:

\begin{subequations}
\begin{align}
\frac{\partial F}{\partial A} = 2\lambda |\phi|^2 A &= 0 \\
\frac{\partial F}{\partial \phi^*} = \phi \left( \frac{a}{2} + \lambda A + \frac{b}{2}|\phi|^2\right) &=0
\end{align}
\end{subequations}

Portanto, obtemos para $A=0$:

\begin{equation}
\phi = sqrt{-\frac{2a}{b}}
\end{equation}

Expandindo ao redor do mínimo($\phi = -\frac{2a}{b} + v(x)$) e mantendo apenas termos de ordem no máximo quadrática, obtemos:

\begin{equation}
\begin{split}
F[\phi(x),A(x)] &= \int \mathbf{d}^d x |\triangledown \phi|^2 + \frac{a}{2} |\phi|^2 + \frac{b}{4} |\phi|^4 + (\triangledown A)^2 + \lambda |\phi|^2 A^2 \\
&= \int \mathbf{d}^d x |\triangledown v|^2 + \frac{a}{2} v^2 + (\triangledown A)^2 + \lambda \sqrt{\frac{-2a}{b}} A^2
\end{split}
\end{equation}

O que desacopla os campos dando massa ao campo $A$.