Nesse problema, temos que minimizar a equação:

\begin{equation}
f(m) = \frac{a}{2} m^2 + \frac{b}{4} m^4 + \frac{c}{6} m^6
\end{equation}

Derivando esta em relação a $m$ e igualando a $0$, obtemos:

\begin{equation}
a m + b m^3 + c m^5 = 0
\end{equation}

Cujas soluções são:

\begin{subequations}
\begin{align}
m &= 0 \\
m &= \sqrt{\frac{-b \pm \sqrt{b^2 - 4ac}}{2c}} \\
m &= -\sqrt{\frac{-b \pm \sqrt{b^2 - 4ac}}{2c}}
\end{align}
\label{sols}
\end{subequations}

Isso implica na condição para as soluções que:

\begin{equation}
a > a_0 =\frac{b^2}{4c}
\end{equation}

Mas temos, também, a condição:

\begin{equation}
\frac{\partial^2 f}{\partial m^2} \geq 0
\label{second_deriv}
\end{equation}

O que leva a:

\begin{equation}
a + 3 b m^{2} + 5 c m^{4} \geq 0
\label{mincond}
\end{equation}


Além disso para que a solução seja de fato um mínimo global, teremos:

\begin{equation}
f(m) < 0
\end{equation}

Para que a solução $m=0$ não seja mínimo global. Dessa maneira, substituímos a solução com sinal adequado, ou seja, positivo dentro da raíz. Assim, obtemos:

\begin{equation}
\begin{split}
a + \frac{b}{2} m^2 + \frac{c}{3} m^4 &< 0 \\
\left( a+bm^2+cm^4 \right) + -\frac{b}{2}m^2 -\frac{2}{3}cm^4 &< 0 \\
-\frac{b}{2} &< \frac{2}{3} cm^2 \\
-\frac{3b}{4c} &< m^2 \\
-\frac{3b}{4c} &< \frac{-b-\sqrt{b^2-4ac}}{2c} \\
-\frac{b}{2} &< \sqrt{b^2-4ac} \\
\frac{b^2}{4} &< b^2 - 4ac \\
a < a_1 &= \frac{3b^2}{16c}
\end{split}
\label{p1_cond}
\end{equation}

A solução $m=0$, combinada com a condição \eqref{second_deriv}, nos leva a:

\begin{equation}
a \geq 0
\end{equation}

Mostrando que é necessário que $a<0$ para que haja esse mínimo local.